\section{Назовите основные группы финансовых показателей}

В современной рыночной экономике, в которой финансы представляют ее основной элемент, анализ принято применять для того, чтобы выявить сущности, закономерности, тенденции и оценки социально-экономических процессов, изучить финансово-хозяйственную деятельности в разных сферах воспроизводства и на всех уровнях.

Для достижения целей финансового анализа необходимо решить определенный взаимосвязанный набор аналитических задач. Аналитическая задача является конкретизацией целей анализа, учитывая организационные и информационные возможности проведения анализа.

Финансовый анализ есть часть общего экономического анализа предприятия, а также часть общего, полного анализа хозяйственной деятельности.

Финансовый анализ, как правило, начинают с расчета финансовых показателей предприятия.

Рассчитываемые показатели объединяют в группы. Финансовые коэффициенты каждой группы характеризуют два аспекта анализа: для целей ликвидации и для целей функционирования предприятия.

В состав показателей каждой группы входят несколько основных общепринятых показателей и множество дополнительных, рассчитываемых в зависимости от целей анализа и управления. Цели анализа могут предполагать как комплексный анализ финансового положения предприятия, так и сравнительно простой анализ на базе основных аналитических финансовых коэффициентов, либо расчет одного из них.

Аналитические финансовые коэффициенты характеризуют соотношения между различными статьями бухгалтерской (финансовой) отчетности. Например, коэффициенты платежеспособности и ликвидности позволяют сравнивать долговые обязательства предприятия с имеющимися у него активами, коэффициент автономии определяет долю собственного капитала в совокупных активах.

В европейских странах и в США применяется практика сравнения коэффициентов со среднеотраслевыми их значениями. Наличие отклонений значений коэффициентов от срёднеотраслевых является предпосылкой для более детального анализа финансового состояния фирмы.

В экономической литературе встречается многообразие терминов для определения по сути одного и того же финансового показателя. В этой связи целесообразно привести наиболее часто встречающиеся их синонимы.

Наиболее распространенными для анализа являются рассчитываемые пять групп финансовых показателей.

В первую группу входят коэффициенты, характеризующие платежеспособность и ликвидность предприятия. Достаточно высокий уровень платежеспособности предприятия является обязательным условием возможности привлечения дополнительных заемных средств и получения кредитов. Кроме того, в эту группу входят показатели, позволяющие судить о возможности предприятия функционировать в дальнейшем. Например, показатель чистого оборотного капитала позволяет судить, насколько предприятие способно погасить свои краткосрочные обязательства и продолжить операционную деятельность.

\[ K_{\text{абс. л.}} = \dfrac{\text{Наиболее ликвидные активы}}{\text{Наиболее срочные обязательства и краткосрочные пассивы}} \]
\[ K_{\text{крит. л.}} = \dfrac{\text{Дебиторская задолженность и прочие активы}}{\text{Наиболее срочные обязательства и краткосрочные пассивы}} \]
\[ K_{\text{тек. л.}} = \dfrac{\text{Cтоимость всех оборотных средств}}{\text{Краткосрочные обязательства}} \]

Во второй группе объединяют показатели финансовой устойчивости. В экономической литературе эту группу называют также показателями структуры капитала и платежеспособности либо коэффициентами управления источниками средств.

\[ K_{\text{авт}} = \dfrac{\text{Источники средств}}{\text{Итог баланса}} \]
\[ K_{\text{соотн. заем. и собств. ср-в}} = \dfrac{\text{Величина обязательств}}{\text{Собственные средства}} \]
\[ K_{\text{обесп. собств. ср-ми}} = \dfrac{\text{Собственные оборотные средств}}{\text{Стоимость запасов и затрат}} \]

В третью группу входят показатели рентабельности.

\[ K_{\text{о. р.}} = \dfrac{\text{Валовая прибыль}}{\text{Стоимость имущества}} \]
\[ K_{\text{ч. р.}} = \dfrac{\text{Чистая прибыль}}{\text{Стоимость имущества}} \]
\[ K_{\text{р. ск}} = \dfrac{\text{Чистая прибыль}}{\text{Величина собственного капитала}} \]
\[ K_{\text{р. пф}} = \dfrac{\text{Валовая прибыль}}{\text{Стоимость основных и оборотных активов}} \]

Четвертую группу представляют показатели деловой активности. Их также называют коэффициентами управления активами.

\[ $$\text{Общ. капиталоотд.} $$= \dfrac{\text{Оборот}}{\text{Средняя стоимость имущества}} \]
\[ $$\text{Отдача осн. произв. ср-в и нем-х акт.}$$ = \dfrac{\text{Оборот}}{\text{Средняя стоимость производственных средств и нематериальных активов}} \]
%\[ K_{\text{р. ск}} = \dfrac{\text{Чистая прибыль}}{\text{Величина собственного капитала}} \]
%\[ K_{\text{р. пф}} = \dfrac{\text{Валовая прибыль}}{\text{Стоимость основных и оборотных активов}} \
Оборачиваемость всех оборотных активов = \dfrac{оборот}{средняя стоимость оборотных активов}
В пятую включают показатели рыночной активности и положения на рынке ценных бумаг.