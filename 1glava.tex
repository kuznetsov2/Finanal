\section{Анализ динамики прибыли}

Анализ финансовых результатов коммерческой организации (предприятия) является, безусловно, одной из важнейших составляющих финансового анализа ее деятельности, направленной на получение прибыли, и основан в значительной степени на данных формы № 2 «Отчет о прибылях и убытках» бухгалтерской (финансовой) отчетности.

Анализ финансовых результатов по данным отчета о прибылях и убытках проводится согласно принципу дедукции и позволяет исследовать их формирование.

Анализ финансовых результатов коммерческой организации начинается с изучения объема, состава, структуры и динамики прибыли (убытка) до налогообложения в разрезе основных источников ее формирования, которыми являются прибыль (убыток) от продаж и прибыль (убыток) от прочей деятельности, т.е. сальдо прочих доходов и расходов.

По итогам расчетов делается вывод о влиянии на отклонение суммы прибыли (убытка) до налогообложения изменений величин источников ее формирования: прибыли (убытка) от продаж и прибыли (убытка) от прочей деятельности.

Поскольку качество прибыли (убытка) до налогообложения определяется ее структурой, то целесообразно обратить особое внимание на изменение удельного веса прибыли от продаж в прибыли до налогообложения. Его снижение рассматривается как негативное явление, свидетельствующее об ухудшении, качества прибыли до налогообложения, так как прибыль от продаж является финансовым результатом от текущей (основной) деятельности предприятия и считается его главным источником средств.

Поэтому желательно следующее соотношение темпа роста прибыли от продаж ($\text{ТР}_{\text{пр}}$) и темпа роста прибыли до налогообложения ($\text{ТР}_{\text{пдн}}$):

$\text{ТР}_{\text{пр}}$ > $\text{ТР}_{\text{пдн}}$

Данное соотношение темпов роста отражает ситуацию, в которой удельный вес прибыли от продаж в прибыли до налогообложения как минимум не уменьшается и, следовательно, качество прибыли до налогообложения по меньшей мере не ухудшается.

Далее анализируются основные источники формирования прибыли (убытка) до налогообложения: прибыль (убыток) от продаж и прибыль (убыток) от прочей деятельности — в отдельности.

Анализ прибыли (убытка) от продаж начинается с изучения ее объема, состава, структуры и динамики в разрезе основных элементов, определяющих ее формирование: выручки (нетто) от продаж, себестоимости продаж, управленческих и коммерческих расходов. При этом в ходе анализа структуры за 100\% берется выручка (нетто) от продаж как наибольший положительный показатель.

По итогам аналитических расчетов делается вывод о влиянии на отклонение прибыли (убытка) от продаж изменений величин каждого из элементов, определяющих ее формирование.

Далее проверяется выполнение условия оптимизации прибыли от продаж:

$$\text{ТР}_{\text{врн}}$$ > $$\text{ТР}_{\text{сп}}$$,

где $\text{ТР}_{\text{врн}}$ — темп роста выручки (нетто) от продаж;

$\text{ТР}_{\text{сп}}$ — темп роста полной себестоимости реализованной продукции (суммы себестоимости продаж, управленческих и коммерческих расходов).

Данное соотношение темпов роста ведет к снижению удельного веса полной себестоимости в выручке (нетто) от продажи, соответственно, к повышению эффективности текущей деятельности коммерческой организации. В случае невыполнения условия оптимизации прибыли от продаж выявляются причины его невыполнения.

Анализ прибыли (убытка) от прочей деятельности ведется в разрезе формирующих ее доходов и расходов. В ходе анализа изучаются ее объем, состав, структура и динамика. При этом структуры доходов и расходов, связанных с прочей деятельностью, анализируются в отдельности.

По итогам аналитических расчетов делается вывод о влиянии изменения суммы связанных с прочей деятельностью доходов и расходов в целом и отдельных их элементов на отклонение величины прибыли (убытка) от прочей деятельности.

Анализ чистой прибыли (убытка) ведется в разрезе определяющих ее элементов, которыми являются прибыль (убыток) до налогообложения, отложенные налоговые активы, отложенные налоговые обязательства и текущий налог на прибыль. В ходе анализа изучаются ее объем, состав, структура и динамика.

По итогам расчетов делается вывод о влиянии на отклонение суммы чистой прибыли (убытка) изменений величин определяющих ее элементов.